%-------------------------------------------------------------------------------------------
% Vorlage erstellt von sli92
% Latex für Einsteiger: http://latex.mschroeder.net/#textformatierung
% Formeln in Latex: http://www.hosi.de/latex/mathe.htm
%-------------------------------------------------------------------------------------------
%PRÄAMBEL
%-------------------------------------------------------------------------------------------

\documentclass[a4paper,14pt,headsepline]{scrartcl}

\usepackage[ngerman]{babel}
\usepackage[utf8]{inputenc}
\usepackage{fancyheadings}
\usepackage{graphicx}
\usepackage{eurosym}

\usepackage{hyperref}

% Absatzeinrückung
%++++++++++++++++++++++++++
%\setlength{\parskip}{5pt}
%\setlength{\parindent}{0pt}

\setlength{\parskip}{1.5em}
\setlength{\parindent}{0pt}

% Kopf- und Fußzeile
%++++++++++++++++++++++++++

\pagestyle{fancy}
\lhead{\bfseries netcon}
%\chead{Lipp}
\rhead{\nouppercase{\leftmark}}

%C für Center
\fancyfoot[C]{ \thepage}

%-------------------------------------------------------------------------------------------
%DOKUMENT
%-------------------------------------------------------------------------------------------

\begin{document}

% Titelseite
%++++++++++++++++++++++++++
\author{Lipp, Pietryka} 
\title{Überblick \linebreak Diplomarbeit: netcon} 
\date{} 
\maketitle



\newpage

\section*{Vorwort}
In vielen Fällen ist bereits eine Infrastruktur vorhanden, sei es ein Firmen- oder Heimnetzwerk auf Ethernet-Basis. Wieso sollte man dieses nicht nutzen, um einfache Steuer- und Messaufgaben zu realisieren? Wieso müssen für einfachste Anwendungen, wie z.B. die Überwachung von Wettergrößen, bereits neue Messsysteme angeschafft werden? Das dafür notwendige Netzwerk, ist häufig bereits vorhanden. Vor allem Privatanwender wünschen sich oft eine kostengünstige Möglichkeit. Netcon hat das Ziel, diesem Bedürfnis nachzukommen und eine einfache Lösung anzubieten. Zudem stehen alle erstellten Entwicklungen unter der OpenSource-Lizenz. 

% Inhaltsverzeichnis
%++++++++++++++++++++++++++
\tableofcontents


%Inhalt
%++++++++++++++++++++++++++
\newpage
\section{Was ist netcon?}

\textbf{Netcon} bezeichnet ein flexibles Mess- und Steuersystem zur Einbindung in ein bestehendes Ethernet-Netzwerk. 

Das System garantiert kein Echtzeitverhalten, weshalb es auch nur für nicht zeitkritische Anwendungen geeignet ist. Grund hierfür ist die Wahl der Kommunikationsschnittstelle, die mit dem Internet Protocol (IP) über Ethernet, zeitkritische Übertragungen nicht sicherstellt. Die Einbindung in ein bestehendes Firmen- oder Heimnetzwerk ist damit aber umso einfacher. Bis auf die Erstellung von netcon-kompatiblen Modulen ist im einfachsten Fall nur ein handelsüblicher Router erforderlich.

Im Gegensatz zu vielen anderen Systemen ist netcon kein Produktpaket, das so im Geschäftsregal stehen soll, sondern vielmehr eine Vereinbarung bzw. Protokoll, woran sich Aktor- oder Sensormodule halten müssen um gemeinsam in einem System zu funktionieren. Dazu stellt die Diplomarbeit Firmware für einige Mikrocontroller-Systeme und eine plattformunabhängige Verwaltungsumgebung zur Verfügung. Im Rahmen dieser Diplomarbeit werden also netzwerkfähige Steuer- und Messmodule geschaffen

\newpage

\section{Grundlegender Aufbau}
Vorausgesetzt das Netzwerk besteht bereits, sind weiters netzwerkfähige Aktor- und Sensormodule, sowie eine Betriebsumgebung für die Verwaltungssoftware notwendig. Diese Umgebung kann jeder Computer sein, der mit einer Java Virtual Runtime und einem http-Webserver mit PHP Unterstützung ausgestattet ist. Die Module müssen sich an die Konventionen halten, die in zwei Protokollen spezifiziert sind. Abb. \ref{netcon_aufbau} zeigt den grundlegenden Aufbau eines solchen Systems. 

\begin{figure}[h]
\begin{center}
\fbox{
	%Rahmengroesse	
	\begin{minipage}{0.7 \paperwidth}
	\begin{center}
	%Bildgroesse
	\includegraphics[width=0.7 \paperwidth]{./bilder/netcon_aufbau.png}
	\caption{Grundlegender Aufbau}
	\label{netcon_aufbau}
	\end{center}
	\end{minipage}
}
\end{center}
\end{figure}

\newpage

Empfehlenswert ist ein DHCP-Server, der sich um die automatische Zuweisung der IP-Adressen im Netzwerk kümmert. Die statische Vergabe ist dennoch möglich. Als Firmenanwender wird dieser DHCP-Server wahrscheinlich ein eigener Computer sein. Privatanwender verwenden im Normalfall einen \linebreak (WLAN)Router. Abb. \ref{netcon_aufbau} ist auf den Einsatz in Unternehmen ausgerichtet. Privatanwender benötigen oftmals keinen Switch, sondern lediglich ihren bereits vorhandenen 4-Port-Router. Dieses Szenario ist in Abb. \ref{netcon_aufbau_privat} verdeutlicht. 

\begin{figure}[h]
\begin{center}
\fbox{
	%Rahmengroesse	
	\begin{minipage}{0.7 \paperwidth}
	\begin{center}
	%Bildgroesse
	\includegraphics[width=0.5 \paperwidth]{./bilder/netcon_aufbau_privat.png}
	\caption{Grundlegender Aufbau für Privatanwender}
	\label{netcon_aufbau_privat}
	\end{center}
	\end{minipage}
}
\end{center}
\end{figure}

Die Verwaltungssoftware läuft entweder separat auf einem Server, oder gemeinsam auf dem Client-Computer, auf dem zur Anzeige der Messdaten und zur Steuerung der Aktormodule eine Website aufgerufen wird. 

\newpage

\section{Systemvoraussetzungen}

Zusammengefasst sind folgende Komponenten erforderlich:

\begin{itemize}

\item DHCP-Server (z.B. handelsüberlicher Router)
\item Switch (falls mehr Ports erforderlich)
\item Computer/Server (für den Betrieb der Verwaltungssoftware)
\item Netzwerkfähige Aktor- oder Sensormodule mit implementierten netcon-Protokoll\footnote{Im Kapitel Hardware werden Mikrocontroller-Kits vorgestellt, für die bereits Firmware zur Verfügung steht.} 
\item Netzwerkkabel für die Verbindung der Komponenten bzw. Access Point bei drahtloser Übertragung

\end{itemize}

Der Computer/Server sollte folgende Anforderungen  erfüllen:

\begin{itemize}

\item 1 GHz CPU
\item 256 MB RAM (nur für netcon)
\item Netzwerkkarte
\item Betriebssystem inkl. Java JRE 6 und PHP-Webserver
\item SSH-Zugang oder Monitor

\end{itemize}

Je nach Wahl des Betriebssystems können die Anforderungen noch höher, oder sogar schwächer ausfallen.


\newpage
\section{Funktionsweise}

\subsection{Server}

\textbf{Netcon Server}, die Verwaltungssoftware besteht aus zwei Teilen (siehe Abb. \ref{netcon_server_aufbau}). Dem Daemon \textbf{netcond}, zuständig für die Überwachung und Verwaltung der einzelnen Module und Schnittstelle zum Webserver \textbf{netcon web}. Dieser stellt das grafische Frontend bereit, über das Anwender mittles Webbrowser einsteigen können, um sich beispielsweise Messdaten anzeigen zu lassen. 

\begin{figure}[h]
\begin{center}
\fbox{
	%Rahmengroesse	
	\begin{minipage}{0.7 \paperwidth}
	\begin{center}
	%Bildgroesse
	\includegraphics[width=0.4 \paperwidth]{./bilder/netcon_server_aufbau.png}
	\caption{Aufbau von netcon Server}
	\label{netcon_server_aufbau}
	\end{center}
	\end{minipage}
}
\end{center}
\end{figure}

\newpage

Der Daemon sucht nach dem Start alle paar Sekunden das Netzwerk nach verbunden Modulen ab und hält diese in einer Liste. Für jedes dieser Module wird ein neuer Programmfaden erzeugt, der ständig Messdaten abfragt und diese speichert, sowie die Steuerung der Module auf Anfrage übernimmt. Zusätzlich startet der Daemon einen weiteren Prozess, die Schnittstelle, über die der Webserver dann Daten abfragen und Steuerinformationen übermitteln kann. Verbindet sich ein Webbrowser zu diesem Webserver, weißt ein PHP-Script den Daemon an, die aktuellen Daten zu übermitteln. Diese werden dann auf der Website angezeigt und ständig aktualisiert, ohne Neuladen der Seite (siehe Abb. \ref{website} ). Genau umgekehrt können auch Steuerdaten übermittelt werden z.B. bei Drücken eines Buttons.\footnote{Noch nicht implementiert} Dieses Prinzip ist nocheinmal in Abb. \ref{netcon_funktionsweise} grafisch verdeutlicht.

\begin{figure}[h]
\begin{center}
\fbox{
	%Rahmengroesse	
	\begin{minipage}{0.8 \paperwidth}
	\begin{center}
	%Bildgroesse
	\includegraphics[width=0.7 \paperwidth]{website.png}
	\caption{netcon Website}
	\label{website}
	\end{center}
	\end{minipage}
}
\end{center}
\end{figure}

Der Webserver kann jeder beliebige http-Server sein, wie z.B. Apache inkl. PHP Integration. Außerdem ist für die Ausführung von netcond eine Java Runtime Environment (JRE) in Version 6 erforderlich. 

\newpage

\begin{figure}[h]
\begin{center}
\fbox{
	%Rahmengroesse	
	\begin{minipage}{0.8 \paperwidth}
	\begin{center}
	%Bildgroesse
	\includegraphics[width=0.7 \paperwidth]{./bilder/netcon_funktionsweise.png}
	\caption{Funktionsweise des Servers}
	\label{netcon_funktionsweise}
	\end{center}
	\end{minipage}
}
\end{center}
\end{figure}


\newpage

\subsection{Module}

Die Module sind Hardware, die mittels Ethernet über das netcon-Protokoll erreichbar, abfragbar und bedienbar sind. Module können bereits alle benötigten Komponenten - Mess- und Netzwerkeinheit - auf einer Platine vereinen (siehe Abb. \ref{modul_aufbau}). Hier erfolgt die Übertragung zwischen den Sensoren und dem Mikroprozessor (uP) meist über Schnittstellen, wie I2C oder 1-Wire, während der Netzwerkcontroller per SPI mit dem uP kommuniziert. Genauso ist es möglich Mess- mit Netzwerkmodulen mittels UART zu verbinden (Abb. \ref{modul_aufbau2}), entweder weil bereits etwaige Messhardware existiert, oder aufgrund der einfachen Implementierung der seriellen Schnittstelle. Dazu wird später auf das im Rahmen der Diplomarbeit entwickelte Netzwerkmodul eingegangen, das als LAN-UART Umsetzer konzipiert wurde um die Erstellung netzwerkfähiger Hardware zu vereinfachen. Messmodule umfassen beispielsweise Sensoren für Temperatur, Luftdruck, oder aber auch Aktoren, wie z.B. Relais oder Displays. Um den Modulen automatisch eine IP-Adresse zuweisen zu können und damit Konfigurationsarbeit zu ersparen, sollten diese auch das DHCP-Protokoll unterstützen. 

\begin{figure}[h]
\begin{center}
\fbox{
	%Rahmengroesse	
	\begin{minipage}{0.8 \paperwidth}
	\begin{center}
	%Bildgroesse
	\includegraphics[width=0.7 \paperwidth]{./bilder/modul_aufbau.png}
	\caption{Aufbau eines Moduls}
	\label{modul_aufbau}
	\end{center}
	\end{minipage}
}
\end{center}
\end{figure}

\newpage

\begin{figure}[h]
\begin{center}
\fbox{
	%Rahmengroesse	
	\begin{minipage}{0.8 \paperwidth}
	\begin{center}
	%Bildgroesse
	\includegraphics[width=0.70 \paperwidth]{./bilder/modul_aufbau2.png}
	\caption{netzwerkfähiges Messmodul mittels LAN-UART Umsetzer}
	\label{modul_aufbau2}
	\end{center}
	\end{minipage}
}
\end{center}
\end{figure}

\begin{itemize}

\item Das \textbf{Modul} ist das Gesamtpaket der netzwerkfähigen Hardware, die später mittels Software verwaltet wird. Es umfasst das Mess- und Netzwerkmodul.
\item \textbf{Messmodule} vereinen Sensoren und Mikroprozessor. 
\item Das \textbf{Netzwerkmodul} macht ein Messmodul netzwerkfähig, damit die Einbindung in ein Ethernet-Netzwerk erst möglich wird.
\item Um die Daten einzelner Sensoren, die zusammen auf einem Modul angebracht sind abfragen zu können, wurde der Begriff \textbf{Device} geschaffen. Jedes Modul kann ein- oder bis zu neun Devices (Sensoren) besitzen, die in der Software dann über ihre Device-ID angesprochen werden können.

\newpage

\section*{Notizen}

\end{itemize}
 
\end{document}